\documentclass[11pt]{article}
\usepackage[margin=1in]{geometry}
\usepackage{amsfonts, amsmath, amssymb}
\usepackage{graphicx}
\usepackage{fancyhdr}
\usepackage[nottoc, notlot, notlof]{tocbibind}
\usepackage{float}
\usepackage{caption}
\usepackage{titlesec}
\PassOptionsToPackage{hyphens}{url}\usepackage{hyperref}
\usepackage{textcomp}

\newcommand{\sectionbreak}{\clearpage}

% FORMATTING CODE 
\usepackage{listings}
\usepackage{color}

\definecolor{dkgreen}{rgb}{0,0.6,0}
\definecolor{gray}{rgb}{0.5,0.5,0.5}
\definecolor{mauve}{rgb}{0.58,0,0.82}

\lstset{frame=tb,
  language=bash,
  aboveskip=3mm,
  belowskip=3mm,
  showstringspaces=false,
  columns=flexible,
  basicstyle={\small\ttfamily},
  numbers=none,
  numberstyle=\tiny\color{gray},
  keywordstyle=\color{blue},
  commentstyle=\color{dkgreen},
  stringstyle=\color{mauve},
  breaklines=true,
  breakatwhitespace=true,
  tabsize=3
}

% CUSTOM HEADER & FOOTER
\pagestyle{fancy}
\fancyhead{}
\fancyfoot{}
\fancyhead[L]{The Direct Daylight Project}
\fancyhead[R]{\today}
\fancyfoot[C]{\thepage}

\parindent 0ex

\renewcommand \labelitemi{--}

\begin{document}
	
\begin{titlepage}
\title{The Direct Daylight Project}
\author{Alex Tan}
\date{\today}
\maketitle{}
\thispagestyle{empty}
\end{titlepage}

\tableofcontents
\thispagestyle{empty}
\clearpage

\setcounter{page}{1}

\section{Git \& SSH}
This section covers how to download the project files, make changes, and sync these changes so that it can be viewed by the whole team.
\subsection{Generate an SSH key}
You will need an SSH key to gain access to the project files. Think of it as like a password. Follow the following steps to get started.
\begin{enumerate}
\item Open a terminal.
\item Run \lstinline{ssh-keygen}.
\begin{itemize}
\item You will be prompted for a location to save the \lstinline{id_rsa.pub} file. Leave this blank to save it in the default location.
\end{itemize}
\item Take note of where the \lstinline{id_rsa.pub} file is stored. You can find the location of the file in the command prompt.
\begin{itemize}
\item On Windows, this is usually \lstinline{C:\Users\your_username/.ssh/id_rsa.pub}
\item On Mac, this is usually \lstinline{/Users/your_username/.ssh/id_rsa.pub}
\item On Linux, this is usually \lstinline{~/.ssh/id_rsa.pub}
\end{itemize}
\item Send the \lstinline{id_rsa.pub} file to Alex.
\end{enumerate}
Note that you will also have a file named \lstinline{id_rsa} in the same directory. This is known as your private key and should \textbf{not} be shared with anyone.
\subsection{Setting up git}
\paragraph{}
Git is a tool for keeping track of changes in project files, and for coordinating work between multiple people across the team. You can think of it as a more sophisticated version of Google Drive, or Microsoft OneDrive.
\paragraph{}
To get started, first check if you have git installed, and download it if it is not installed.
\begin{itemize}
\item On \textbf{Windows}, open the Start Menu and search for git. Unless you have used git before, it probably will not be installed. Visit \url{https://git-scm.com/download/win} to download and install git.\\\\
To check if it installed correctly, open the Start Menu and search for git. After opening git, a command prompt should open. Now run \lstinline{git --version}.
\item On \textbf{Mac}, you may or may not have git installed. You can check by opening a terminal and running \lstinline{git --version}. If git is not installed, visit \url{https://git-scm.com/download/mac} to download and install git.\\\\
To check if it is installed correctly, open a terminal, and run \lstinline{git --version}.
\item On \textbf{Linux}, you may or may not have git installed. You can check by opening a terminal and running \lstinline{git --version}. If git is not installed, visit \url{https://git-scm.com/download/linux} and follow the instructions.\\\\
To check if it is installed correctly, open a terminal, and run \lstinline{git --version}.
\end{itemize}
\subsection{Command prompt basics}
There are numerous commands that can be run in the command prompt. The following are the most useful to get you started.
\begin{center}
\begin{tabular}{ll}
\lstinline!ls! & Lists all the files and directories in your current directory.\\ 
\lstinline!pwd! & Prints the location of your current directory.\\
\lstinline!cd directory_name! & Navigates to the specified directory (relative to where you currently are).\\
\lstinline!cd .. ! & Navigates to the parent directory (relative to where you currently are).\\
\end{tabular} 
\end{center}
\subsection{Cloning the repo}
We first need to download the project files. In the git command line, navigate to somewhere you want to download the project to, and then run the following command.
\begin{lstlisting}
git clone git@directdaylight.com:direct-daylight-game
\end{lstlisting}
This will create a new folder named \lstinline{direct-daylight-game} and download the project into that folder.
\subsection{Git basics}
Now that you have a copy of the project files, you can start editing the files. The following section covers how to \textbf{commit} and \textbf{push }those changes so that it can be viewed by everyone on the team.\\\\

\begin{enumerate}
\item \lstinline{git add .}
\item If this is your first time using git you will also need to run the following commands. Replace the email and name with your real email and name.
\begin{enumerate}
\item git config --global user.email username@email.com''
\item git config --global user.name "name"
\end{enumerate}
\item \lstinline{git commit -m "commit message"}
\item \lstinline{git push origin master}
\end{enumerate}
To pull changes, run \lstinline{git pull origin master}

\section{Trello}
Trello is an app used for organising and planning out tasks. At each weekly meeting, tasks will be reviewed and assigned.
\begin{itemize}
\item Sign up for Trello at \url{https://trello.com}
\item Accept the invitation to the Trello board at \\ \url{https://trello.com/invite/b/CW6IRbz0/b1605676bf57e5327f3cd156d03979d7/direct-daylight-game}
\item You can now view and edit the Trello board at \url{https://trello.com/b/CW6IRbz0}
\end{itemize}
\end{document}